A continuación pasaremos en limpio las conclusiones que pudimos sacar con la realización del presente trabajo práctico. Dado que en la práctica las tres áreas de experimentación sobre las que realizamos nuestros análisis se entrelazan y son interdependientes, trataremos de que las conclusiones sean globales y no un mero enlistamiento de los resultados de cada sección. 

Vimos que en general interpolar con el método de vecino más cercano nunca da resultados mejores en términos de calidad que los otros métodos, pues carece de toda fluidez y entre más cuadros se desean agregar más notoriamente impacta esto. Si bien puede resultar ser el más rápido, la diferencia con el costo de usar interpolación lineal que presenta resultados más satisfactorios, no es demasiado grande, por lo que este casi siempre va a terminar siendo preferible. 

Entre la interpolación lineal y las tres variantes de splines cúbicos no se encontraron grandes diferencias de comportamiento en cuanto a la calidad de los videos producidos. Lineal tiende a ser un poco más sólida en los casos en que la cámara se mueve muy rápido pues depende de los frames inmediatos a interpolar (y por tanto que menos difieren), en contraposición a splines que requiere cuadros más alejados (lo que termina convirtiéndose en una \emph{desinformación} para el método). En esa línea, considerar bloques de tamaño 4 u 8 para splines resulta más robusto que tomar 12. Pero a su vez, los tiempos de splines con bloques de tamaño 4 aumentan tan vertiginosamente (especialmente en base a la cantidad de cuadros originales del video y su resolución) que en general no tiene sentido siquiera considerar esta opción.

Por lo tanto, de las tres variantes de splines la más razonable parece ser la que considera bloques de tamaño 8. Para videos donde no hay movimientos de cámara rápidos y los objetos se mueven a una velocidad normal (y aún habiendo algunos movimientos bruscos) este método da resultados un poco más fluidos que los de la interpolación lineal. No obstante el tiempo de cómputo a pagar sigue siendo sensiblemente más alto. Por lo tanto, debe escogerse el método que mejor ajuste a la situación: si hay movimientos abruptos de cámara o no, si se desea preprocesar el video del lado del servidor o si se quiere procesar durante el \textit{streaming} del lado del usuario.

Desde el punto de vista práctico propuesto en el trabajo, el cual proponía intentar pasar a cámara lenta videos normales para evitar sobrecargar una red con videos mas pesados de lo necesario, llegamos a la conclusión de que esto es posible aunque tiene ciertas limitaciones, ya que excepto para videos con cambios muy graduales en los valores de sus pixels, los métodos utilizados comienzan a presentar grandes falencias al intentar interpolar de a mas de 2 cuadros, a partir de esos valores no se considera que los videos resultantes tengan una calidad suficientemente buena para el contexto planteado, estos estarían demasiado alejados de una versión real de cámara lenta.

Como último punto a destacar, vimos que ninguno de nuestros métodos tolera bien los cambios de tomas. Llegamos a la conclusión de que lo mejor que se puede hacer ante esta situación es subdividir el video en las distintas tomas para luego interpolar con un método u otro cada subdivisión de manera independiente. El como implementar esto de manera eficiente queda como trabajo a futuro, aunque una primera aproximación sencilla sería simplemente ir recorriendo el video frame por frame, tomando la diferencia respecto del anterior (usando alguna métrica adecuada), y a partir de estos valores armar la partición de tomas.
